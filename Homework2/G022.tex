
% RUN:
% pdflatex -output-directory=/Users/salvatorpes/Desktop/Aprendizagem/Homework2/trash /Users/salvatorpes/Desktop/Aprendizagem/Homework2/G022.tex

% ir a settings.json e adicionar:
% // According to the wiki, the string latex-workshop.latex.autoBuild.run has three possible values: never, onSave and onFileChange(default).
% "latex-workshop.latex.autoBuild.run": "never",

\documentclass{article}

\author{Pedro Curvo (ist1102716) $|$ Salvador Torpes (ist1102474)}

\usepackage[utf8]{inputenc}
\usepackage[english]{babel}
% \usepackage[letterpaper,top=10mm,bottom=15mm,left=15mm,right=15mm,marginparwidth=1.75cm]{geometry}
% \usepackage[letterpaper,top=10mm,bottom=15mm,left=15mm,right=15mm,marginparwidth=1.75cm]{geometry}
\usepackage[letterpaper,margin=1in,marginparwidth=1.75cm]{geometry}
\usepackage{multicol}
\usepackage{biblatex}
\addbibresource{Bibliografia.bib}
\usepackage{graphicx}
% \graphicspath{{../Homework1/images/}}
\usepackage{subcaption}
\usepackage{tabularx}
\usepackage{booktabs}
\usepackage{array}
\usepackage{makecell}
\usepackage{multirow}
\usepackage{amsmath}
\usepackage{makecell}
\usepackage{url}
\usepackage{csquotes}
\usepackage{caption}
\usepackage{enumitem}
\usepackage{textcomp}
\usepackage{pdflscape}
\usepackage{makeidx}
\usepackage{amsmath}
% \usepackage{tocbibind}
\providecommand{\tightlist}{\relax}
\usepackage{tocloft}
\renewcommand{\cftsecindent}{0em}
\renewcommand{\cftsubsecindent}{1em}
\renewcommand{\cftsecfont}{\bfseries}
\renewcommand{\cftsubsecfont}{\itshape}
\setlength{\cftsubsecnumwidth}{0em}

\usepackage[version=4]{mhchem}
\usepackage{hyperref} % Remove "pdftex" option here
\usepackage{float}
\usepackage{fancyhdr}
\usepackage{ragged2e}
\usepackage{xkeyval}
%\usepackage{minted}
%\usemintedstyle{manni}
\usepackage{listings}
\usepackage{amssymb}




\usepackage{tikz}
\usetikzlibrary{positioning}
\usetikzlibrary{positioning, arrows.meta}
\usepackage{adjustbox}
\usepackage{sidecap}



% \usepackage[table,xcdraw]{xcolor}
\usepackage[LY1]{fontenc}
\usepackage{tikz-3dplot}
% \usepackage{pgfplots}
\usetikzlibrary{calc, 3d, arrows}
\usepackage{forest}




\usetikzlibrary{shapes.geometric, arrows}


\lstset{
    language=Python,
    basicstyle=\ttfamily,
    keywordstyle=\color{blue},
    commentstyle=\color{gray},
    stringstyle=\color{orange},
    numbers=left,
    numberstyle=\tiny,
    numbersep=5pt,
    showspaces=false,
    showstringspaces=false,
    breaklines=true,
    frame=tb,
    framexleftmargin=2em,
    xleftmargin=2em,
}


%\usepackage{fontspec}

%\setmonofont{Fira Code}

\fancyhf{}
\cfoot{\thepage}
\fancyhf{} % Clear all header and footer fields
\renewcommand{\headrulewidth}{0pt} % Remove the header rule line
\cfoot{\thepage} % Set the page number in the center of the footer

\pagestyle{fancy} % Apply the fancy page style

\setlength\columnsep{20pt}

\renewcommand{\familydefault}{\sfdefault}

\newenvironment{Figure}
  {\par\medskip\noindent\minipage{\linewidth}}
  {\endminipage\par\medskip}

\makeatletter
\newenvironment{figurehere}
{\def\@captype{figure}}
{}
\makeatother

\hypersetup{
  colorlinks,
  linkcolor=blue,
  anchorcolor=black,
  citecolor=cyan,
  filecolor=cyan,
  menucolor=cyan,
  urlcolor=cyan,
  bookmarksopen=true,
  bookmarksnumbered=true
}

\makeindex


\title{\vspace{-6mm}\includegraphics[width=15mm,scale=2]{images/IST_Logo.png}\\ \vspace{5mm}
Machine Learning - Homework 2 \vspace{-5mm}}
\date{1st Term - 23/24}

\usepackage{sansmathfonts}
\usepackage[T1]{fontenc}
\usepackage[OT1]{fontenc}

\usepackage{listings}
\usepackage{xcolor}

\definecolor{codegreen}{rgb}{0,0.6,0}
\definecolor{codegray}{rgb}{0.5,0.5,0.5}
\definecolor{codepurple}{rgb}{0.58,0,0.82}
\definecolor{backcolour}{rgb}{0.95,0.95,0.92}

\lstdefinestyle{mystyle}{
  backgroundcolor=\color{backcolour},   
  commentstyle=\color{codegray},
  keywordstyle=\color{magenta},
  numberstyle=\tiny\color{codegray},
  stringstyle=\color{codegreen},
  keywordstyle=[2]{\color{orange}},
  keywords=[2]{plt.},
  basicstyle=\ttfamily\footnotesize,
  breakatwhitespace=false,         
  breaklines=true,                 
  captionpos=b,                    
  keepspaces=true,                 
  numbers=left,                    
  numbersep=5pt,                  
  showspaces=false,                
  showstringspaces=false,
  showtabs=false,                  
  tabsize=2,
  frame=single,
  framesep=2pt,
  framerule=0pt,
  xleftmargin=2pt,
  xrightmargin=2pt,
  aboveskip=1em,
  belowskip=1em,
  abovecaptionskip=0.5em,
  belowcaptionskip=0.5em,
  caption=\lstname,
  captionpos=b,
  language=Python,
  morekeywords={as},
  deletekeywords={None},
  emph={self},
  emphstyle=\color{blue},
  escapeinside={(*@}{@*)},
  literate={+}{{\textcolor{blue}{+}}}1
       {*}{{\textcolor{blue}{*}}}1
       {-}{{\textcolor{blue}{-}}}1
       {/}{{\textcolor{blue}{/}}}1
       {=}{{\textcolor{blue}{=}}}1
       {>}{{\textcolor{blue}{>}}}1
       {<}{{\textcolor{blue}{<}}}1
       {==}{{\textcolor{blue}{==}}}2
       {!=}{{\textcolor{blue}{!=}}}2
       {<=}{{\textcolor{blue}{<=}}}2
       {>=}{{\textcolor{blue}{>=}}}2,
  }
    
    \lstset{style=mystyle}
    \usepackage{fancyhdr}
    
    % Define header and footer styles
    \fancypagestyle{plain}{%
      \fancyhf{}% Clear header/footer
      \fancyhead[L]{Homework 2}% Header left
      \fancyhead[C]{2023/2024}% Header left
      \fancyhead[R]{Aprendizagem}% Header right
      \fancyfoot[C]{\thepage}% Footer center
      \renewcommand{\headrulewidth}{0.4pt}% Header rule
      \renewcommand{\footrulewidth}{0pt}% Footer rule
    }
    
    % Apply the style to all pages except the first one
    \pagestyle{plain}
    \thispagestyle{empty} % Remove header/footer from first page
    
\begin{document}
    
\renewcommand{\arraystretch}{1.7}
\setlength{\columnseprule}{0.4pt}
\tdplotsetmaincoords{70}{110} % Set the viewing angle
\newcolumntype{M}[1]{>{\centering\arraybackslash\vspace{#1}}m{0.5\linewidth}<{\vspace{#1}}}
\newcolumntype{C}[2]{>{\centering\arraybackslash\vspace{#1}\rule{0pt}{#1}\hspace{0pt}}m{#2}}
\ifx\undefined\w
\newcolumntype{w}[1]{>{\centering\arraybackslash}m{#1}}
\fi
\renewcommand*\familydefault{\sfdefault} %% Only if the base font of the document is to be sans serif
\maketitle
\vspace{-5mm}
\hrulefill





\section*{Dataset}

The following dataset will be used for this homework:

\begin{table}[h!]
\centering
\label{tab:dataset1}
\begin{tabular}{|cc|ccccc|c|}
\hline
\multicolumn{2}{|c|}{\multirow{2}{*}{$D$}}                           & \multicolumn{5}{c|}{Input}                                                                                                & \multicolumn{1}{l|}{Output} \\ \cline{3-8} 
\multicolumn{2}{|c|}{}                                               & \multicolumn{1}{c|}{$y_1$} & \multicolumn{1}{c|}{$y_2$} & \multicolumn{1}{c|}{$y_3$} & \multicolumn{1}{c|}{$y_4$} & $y_5$ & $y_6$                       \\ \hline
\multicolumn{1}{|c|}{\multirow{7}{*}{Training Observations}} & $x_1$ & \multicolumn{1}{c|}{0.24}  & \multicolumn{1}{c|}{0.36}  & \multicolumn{1}{c|}{1}     & \multicolumn{1}{c|}{1}     & 0     & A                           \\ \cline{2-8} 
\multicolumn{1}{|c|}{}                                       & $x_2$ & \multicolumn{1}{c|}{0.16}  & \multicolumn{1}{c|}{0.48}  & \multicolumn{1}{c|}{1}     & \multicolumn{1}{c|}{1}     & 0     & A                           \\ \cline{2-8} 
\multicolumn{1}{|c|}{}                                       & $x_3$ & \multicolumn{1}{c|}{0.32}  & \multicolumn{1}{c|}{0.72}  & \multicolumn{1}{c|}{0}     & \multicolumn{1}{c|}{1}     & 2     & A                           \\ \cline{2-8} 
\multicolumn{1}{|c|}{}                                       & $x_4$ & \multicolumn{1}{c|}{0.54}  & \multicolumn{1}{c|}{0.11}  & \multicolumn{1}{c|}{0}     & \multicolumn{1}{c|}{0}     & 1     & B                           \\ \cline{2-8} 
\multicolumn{1}{|c|}{}                                       & $x_5$ & \multicolumn{1}{c|}{0.66}  & \multicolumn{1}{c|}{0.39}  & \multicolumn{1}{c|}{0}     & \multicolumn{1}{c|}{0}     & 0     & B                           \\ \cline{2-8} 
\multicolumn{1}{|c|}{}                                       & $x_6$ & \multicolumn{1}{c|}{0.76}  & \multicolumn{1}{c|}{0.28}  & \multicolumn{1}{c|}{1}     & \multicolumn{1}{c|}{0}     & 2     & B                           \\ \cline{2-8} 
\multicolumn{1}{|c|}{}                                       & $x_7$ & \multicolumn{1}{c|}{0.41}  & \multicolumn{1}{c|}{0.53}  & \multicolumn{1}{c|}{0}     & \multicolumn{1}{c|}{1}     & 1     & B                           \\ \hline
\multicolumn{1}{|c|}{\multirow{2}{*}{Testing Observations}}  & $x_8$ & \multicolumn{1}{c|}{0.38}  & \multicolumn{1}{c|}{0.52}  & \multicolumn{1}{c|}{0}     & \multicolumn{1}{c|}{1}     & 0     & A                           \\ \cline{2-8} 
\multicolumn{1}{|c|}{}                                       & $x_9$ & \multicolumn{1}{c|}{0.42}  & \multicolumn{1}{c|}{0.59}  & \multicolumn{1}{c|}{0}     & \multicolumn{1}{c|}{1}     & 1     & B                           \\ \hline
\end{tabular}
\caption{Dataset}
\end{table}

\newpage

\section*{1\textsuperscript{st} Question}

In order to build the Bayesian classifier for this dataset, we need to compute the class conditional distributions of $\{y_1,y_2\}$, $\{y_3,y_4\}$ and $y_5$, which are the groups of independent input variables of our dataset as well as the priors.

\paragraph{Priors}

First of all, we will compute the priors $P(y_6=A)$ and $P(y_6=B)$:

\begin{align*}
  P(y_6=A) &= \frac{3}{7} \\
  P(y_6=B) &= \frac{4}{7} 
\end{align*}


\paragraph{Distribution of $y_1$ and $y_2$}

We are told that $y_1 \times y_2 \in \mathbb{R}$ follows a normal 2D distribution.
A multivariate normal distribution of $m$ variables $\vec{x} = \{x_1, x_2, ..., x_m\}$ is defined by its mean vector $\vec{\mu}$ and its covariance matrix $\Sigma$:

\[
  P(\vec{x}| \vec{\mu}, \Sigma) = \frac{1}{\sqrt{(2\pi)^m |\Sigma|}} \exp \left( -\frac{1}{2} (\vec{x} - \vec{\mu})^T \cdot \Sigma^{-1} \cdot (\vec{x} - \vec{\mu}) \right)
  \]

In our case, we have $m = 2$, $\vec{x} = \{y_1, y_2\}$ and we need to compute two class conditional distributions $p(\vec{x}|y_6=A)$ and $p(\vec{x}|y_6=B)$.

\paragraph{Distribution of $\{y_1,y_2\}$ given $y_6=A$}
\paragraph{}

Considering the training data in table \ref*{tab:dataset1} with class $y_6=A$, we can compute the mean vector $\vec{\mu}$ and the covariance matrix $\Sigma$ as follows:

\[
  \vec{\mu} =  \left[\begin{matrix} \mu_{y_1} \\ \mu_{y_2} \end{matrix} \right]= \frac{1}{3} \cdot 
  \left[\begin{matrix}
    0.24 + 0.16 + 0.32 \\
    0.36 + 0.48 + 0.72
  \end{matrix}\right] = \left[\begin{matrix}
    0.24 \\
    0.52
  \end{matrix}\right] \\
\]

\[
  \Sigma = \left[ \begin{matrix}
    \sigma_{y_1}^2 & \sigma_{y_1,y_2} \\
    \sigma_{y_1,y_2} & \sigma_{y_2}^2
  \end{matrix} \right] = \frac{1}{3} \cdot \begin{bmatrix}
    \sum_{i=1}^{3} (y_{1i} - \mu_{y_1})^2 & \sum_{i=1}^{3} (y_{1i} - \mu_{y_1})(y_{2i} - \mu_{y_2}) \\
    \sum_{i=1}^{3} (y_{1i} - \mu_{y_1})(y_{2i} - \mu_{y_2}) & \sum_{i=1}^{3} (y_{2i} - \mu_{y_2})^2
  \end{bmatrix}  = \begin{bmatrix}
    0.0064 & 0.0064 \\
    0.0064 & 0.0064
  \end{bmatrix}
\]

Now we need to compute both $|\Sigma|$ and $\Sigma^{-1}$:

\[
  |\Sigma| = 
\]

\[
  \Sigma^{-1} = 
\]

Therefore, we have the normal distribution of $\{y_1,y_2\}$ given $y_6=A$:

\[
    P((y_1,y_2)|y_6=A) = \frac{1}{\sqrt{(2\pi)^2 |\Sigma|}} \exp \left( -\frac{1}{2} ((y_1,y_2) - \vec{\mu})^T \cdot \Sigma^{-1} \cdot ((y_1,y_2) - \vec{\mu}) \right)=
\]





\paragraph{Distribution of $y_3$ and $y_4$}

The class conditional distributions of $y_3$ and $y_4$ come directly from the information in table \ref{tab:dataset1} and they are given by:

\begin{table}[h!]
\centering
\begin{tabular}{|cc|cc|}
\hline
\multicolumn{2}{|c|}{\multirow{2}{*}{$P(y_3 \cap y_4|y_6=A)$}} & \multicolumn{2}{c|}{$y_3$}                     \\ \cline{3-4} 
\multicolumn{2}{|c|}{}                                   & \multicolumn{1}{c|}{0}           & 1           \\ \hline
\multicolumn{1}{|c|}{\multirow{2}{*}{$y_4$}}     & 0     & \multicolumn{1}{c|}{$P(y_3=0 \cap y_4=0|y_6=A) =0$} & $P(y_3=1 \cap y_4=0|y_6=A) =0$ \\ \cline{2-4} 
\multicolumn{1}{|c|}{}                           & 1     & \multicolumn{1}{c|}{$P(y_3=0 \cap y_4=1|y_6=A) =\frac{1}{3}$} & $P(y_3=1 \cap y_4=1|y_6=A) =\frac{2}{3}$ \\ \hline
\end{tabular}
\caption{Distribution of $y_3$ and $y_4$ given $y_6=A$}
\end{table}

\begin{table}[h!]
\centering
\begin{tabular}{|cc|cc|}
\hline
\multicolumn{2}{|c|}{\multirow{2}{*}{$P(y_3 \cap y_4|y_6=B)$}} & \multicolumn{2}{c|}{$y_3$}                     \\ \cline{3-4}
\multicolumn{2}{|c|}{}                                   & \multicolumn{1}{c|}{0}           & 1           \\ \hline
\multicolumn{1}{|c|}{\multirow{2}{*}{$y_4$}}     & 0     & \multicolumn{1}{c|}{$P(y_3=0 \cap y_4=0|y_6=B) = \frac{1}{2}$} & $P(y_3=1 \cap y_4=0|y_6=B) =\frac{1}{4}$ \\ \cline{2-4} 
\multicolumn{1}{|c|}{}                           & 1     & \multicolumn{1}{c|}{$P(y_3=0 \cap y_4=1|y_6=B) =\frac{1}{4}$} & $P(y_3=1 \cap y_4=1|y_6=B) =0$ \\ \hline
\end{tabular}
\caption{Distribution of $y_3$ and $y_4$ given $y_6=B$}
\end{table}



\paragraph{Distribution of $y_5$}

The class conditional distribution of $y_5$ is given by:

\begin{table}[h!]
\centering
\begin{tabular}{|cc|ccc|}
\hline
\multicolumn{2}{|c|}{\multirow{2}{*}{$P(y_5|y_6)$}} & \multicolumn{3}{c|}{$y_5$}                                                                                                        \\ \cline{3-5} 
\multicolumn{2}{|c|}{}                              & \multicolumn{1}{c|}{0}                           & \multicolumn{1}{c|}{1}                           & 2                           \\ \hline
\multicolumn{1}{|c|}{\multirow{2}{*}{$y_6$}}   & A  & \multicolumn{1}{c|}{$P(y_5=0|y_6=A) =\frac{2}{3}$} & \multicolumn{1}{c|}{$P(y_5=1|y_6=A) =0$} & $P(y_5=2|y_6=A) =\frac{1}{3}$ \\ \cline{2-5} 
\multicolumn{1}{|c|}{}                         & B  & \multicolumn{1}{c|}{$P(y_5=0|y_6=B) =\frac{1}{4}$} & \multicolumn{1}{c|}{$P(y_5=1|y_6=B) =\frac{1}{2}$} & $P(y_5=2|y_6=B) =\frac{1}{4}$ \\ \hline
\end{tabular}
\caption{Distribution of $y_5$ given $y_6$}
\end{table}

\section*{2\textsuperscript{nd} Question}

In order to classify the testing observations, we will need to compute the posterior probabilities.




\end{document}